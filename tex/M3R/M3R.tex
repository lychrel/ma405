\documentclass{article}
\usepackage[margin=1in]{geometry}
\usepackage[utf8]{inputenc}
\usepackage[T1]{fontenc}
\usepackage{microtype}

\usepackage{amssymb}
\usepackage{amsfonts}
\usepackage{amsmath}
\usepackage{amsthm}

\usepackage{todonotes}
\usepackage{graphicx}
\usepackage{showlabels}

\usepackage[english]{babel}
\usepackage{blindtext}
% \Blindtext to use.

\RequirePackage[l2tabu, orthodox]{nag}
\usepackage[all,warning]{onlyamsmath}

\usepackage{hyperref}

% blue text for questions
\newcommand{\bt}[1]{\textcolor{blue}{#1}}

\newcommand{\prb}[1]{
	\begin{center}
		\includegraphics[width=\textwidth]{{#1}}
	\end{center}}

\title{M3 RVW}
\author{Jack Lynch}

\begin{document}
	\maketitle
	
	\prb{12}
		This follows from the definition of a linear
		transformations:
		
		\begin{equation*}
			A(cx+dy) = c(Ax) + d(Ay)
		\end{equation*}	
		\noindent or, in our case,
		\begin{equation*}
			A^{-1}(x+y) = A^{-1}x + A^{-1}y
		\end{equation*}
		\noindent But I'm guessing that's not what this means.
		
		Does it have to do with the (assumed) vectors $x$ and
		$y$ being linearly independent?
	
	\pr{17}
		
		
	
\end{document}